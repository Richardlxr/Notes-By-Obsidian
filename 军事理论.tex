\documentclass[UTF8, a4paper, 11pt]{ctexart}

% --- 核心宏包:解决报错的关键 ---
\usepackage{xeCJKfntef}             % 替代 soul,专门处理中文下划线/高亮,永不报错
\usepackage[margin=2.5cm]{geometry} % 页边距
\usepackage{enumitem}              % 优化列表
\usepackage{titlesec}              % 标题格式定制
\usepackage{amsmath}               % 数学公式支持
\usepackage{setspace}              % 行间距控制
\usepackage{xcolor}                % 颜色支持

% --- 自定义命令与样式设置 ---
\setstretch{1.3}                   % 设置 1.3 倍行距
\newcommand{\hr}{\noindent\rule{\textwidth}{0.5pt}\par\vspace{0.5em}} % 分割线

% 将原 Markdown 的 == 映射为中文下划线强调
\newcommand{\强调}[1]{\CJKunderline[thickness=0.6pt]{#1}}

% 调整列表间距
\setlist{nosep, leftmargin=2em, itemsep=2pt}

% 设置标题格式
\titleformat{\section}{\Large\bfseries\raggedright}{}{0em}{}
\titleformat{\subsection}{\large\bfseries}{}{0em}{【\thesubsection】}
\titleformat{\subsubsection}{\normalsize\bfseries}{}{0em}{}

% --- 文档开始 ---
\begin{document}

\section{第一章 \quad 中国国防}

\subsection{第一节、国防概述}

\subsubsection{国防的含义}
国防是国家为抵御外来侵略与颠覆,捍卫国家主权、领土完整、维护国家安全、统一和发展,而进行的军事及与军事有关的政治、经济、科技、文化、教育、外交等方面的活动

\subsubsection{现代国防的基本类型}
\begin{enumerate}
    \item \textbf{扩张型}
    \item \textbf{自卫型}
    \item \textbf{联盟型}:如 \textbf{“美日军事联盟”,“北约”}
    \item \textbf{中立型}
\end{enumerate}

\subsubsection{现代国防的基本特征}
\begin{enumerate}
    \item \textbf{职能的双重性}
    \item \textbf{对抗的整体性}
    \item \textbf{目标的层次性}
    \item \textbf{手段的灵活性}
\end{enumerate}

\subsubsection{古代的兵役制度}
\begin{enumerate}
    \item \textbf{秦汉时期的征兵制}
    \item \textbf{三国两晋南北朝时期的世兵制}
    \item \textbf{隋朝时期的府兵制}
    \item \textbf{宋朝的募兵制}
    \item \textbf{明朝的卫所兵役制}
\end{enumerate}

\subsubsection{新时代防御性国防政策}
\begin{enumerate}
    \item \textbf{坚决捍卫国家主权、安全、发展利益}
    \item \textbf{坚持永不称霸、永不扩张、永不谋求势力范围}
    \item \textbf{贯彻落实新时代军事战略方针}
    \item \textbf{坚持走中国特色强军之路}
    \item \textbf{服务构建人类命运共同体}
\end{enumerate}

\hr

\subsection{第二节、国防建设}

\subsubsection{国防建设的领导体制}
\begin{itemize}
    \item \textbf{中共中央的国防领导职权}
    \begin{enumerate}
        \item 中国共产党是中国特色社会主义事业的领导核心。
        \item 中国人民解放军必须置于中国共产党的绝对领导之下,其最高领导权和指挥权属于中国共产党中央委员会和中央军事委员会。
    \end{enumerate}
    \item \textbf{全国人民代表大会及其常务委员会}
    \begin{itemize}
        \item \textbf{人大}
        \begin{itemize}
            \item \textbf{决定战争与和平问题(决定战争状态)}
            \item 制定有关国防方面的基本法律
            \item 选举中央军事委员会主席 \textbf{(选举、罢免)}
            \item 审查和批准各类报告
            \item 改变常委的决定
        \end{itemize}
        \item \textbf{常委(有点类似)}
        \begin{itemize}
            \item 决定战争状态的宣布、决定全国总动员或者局部动员、制定有关国防方面的法律、审查和批准各类报告
            \item \textbf{监督中央军事委员会工作}
            \item 任免、规定衔级制度、决定授予荣誉称号.etc
        \end{itemize}
    \end{itemize}
    \item \textbf{国家主席}
    \begin{enumerate}
        \item 根据人大决定,\textbf{宣布战争状态}、\textbf{发布动员令}
        \item 公布人大制定的法律、授予勋章、荣誉称号、批准和废除同外国缔结的国防相关条约
    \end{enumerate}
    \item \textbf{国务院 (领导和管理)}
    \begin{enumerate}
        \item \textbf{编制国防建设发展规划和计划}
        \item 制定国防相关方针、政策、行政法规
        \item 领导管理国防科研生产
        \item 管理国防军费、资产
        \item 退役军人安置、国防教育、征兵.ect
    \end{enumerate}
\end{itemize}

\subsubsection{国防建设的新成就}
\begin{itemize}
    \item \textbf{关于军队领导管理体制的新突破}
    \begin{itemize}
        \item 着眼于贯彻新形势下政治建军的要求,推进领导掌握部队和高效指挥部队有机统一,形成\textbf{军委管总、战区主战、军种主建的格局}
        \item 在部队领导管理上形成\textbf{军委──军种──部队的领导管理体系}
        \item 在联合作战指挥上实行\textbf{军委──战区──部队的作战指挥体系}
    \end{itemize}
    \item \textbf{关于高素质新型军事人才(资料窗)}
    \begin{enumerate}
        \item \textbf{军队院校教育}:重在固本强基
        \item \textbf{部队训练实践}:重在转化运用
        \item \textbf{军事职业教育}:重在补充拓展
    \end{enumerate}
\end{itemize}

\hr

\subsection{第三节、武装力量}

\subsubsection{武装力量的组织机构}
\paragraph{1. 中央军委机关(对比以下两者的区别)}
\begin{itemize}
    \item \textbf{军委办公厅}
    \begin{itemize}
        \item \textbf{是中央军委的重要办事机构和执行机关}
        \item \textbf{落实首长指示、组织协调军委各职能部门有序运行(组织管理)}
        \item 其制发的规范性文件具有\textbf{军事规章的效力}
    \end{itemize}
    \item \textbf{军委联合参谋部}
    \begin{itemize}
        \item \textbf{作战筹划、指挥控制、作战指挥保障}
        \item \textbf{研究拟制军事战略和军事需求}
        \item \textbf{组织作战能力评估,组织指导联合训练、战备建设、日常战备}
    \end{itemize}
\end{itemize}

\paragraph{2. 军兵种}
\begin{itemize}
    \item \textbf{陆军、海军、空军、火箭军四大军种}
    \item \textbf{军事航天部队、网络空间部队、信息支援部队、联勤保障部队四大兵种}
\end{itemize}

\subsubsection{中国人民解放军}
\begin{itemize}
    \item 创建于1927.8.1\textbf{南昌起义}
    \item 是\textbf{中国共产党缔造和领导的人民军队、是我国武装力量的主体}
    \item 中国人民解放军由\textbf{现役部队和预备役部队组成}
\end{itemize}

\subsubsection{中国人民武装警察部队}
\begin{enumerate}
    \item 武装警察部队的编制体制
    \begin{itemize}
        \item \textbf{由党中央、中央军委集中统一领导,实行中央军委──武警部队──部队领导的指挥体制}
        \item 武装警察部队执行中国人民解放军的有关命令条例规定,但是\textbf{\强调{武装警察部队和中国人民解放军是并列关系,不是从属!}}
    \end{itemize}
    \item 武装警察部队的职能
    \begin{enumerate}
        \item \textbf{海上维权执法},维护国家主权尊严
        \item \textbf{预防和镇压敌对势力的破坏,应对紧急情况、维护社会治安}
        \item \textbf{执行维稳和反恐任务,保证人民生命财产安全}
        \item \textbf{防卫作战}
    \end{enumerate}
\end{enumerate}

\hr

\subsection{第四节、国防动员}
\paragraph{分类(总、局部动员)}
\begin{itemize}
    \item \textbf{问:施行总动员还是局部动员由什么决定?}
    \begin{itemize}
        \item \textbf{\强调{战争规模、国家战略意图}}
    \end{itemize}
    \item \textbf{问:这俩可以互相转化吗?}
    \begin{itemize}
        \item \textbf{可以}
    \end{itemize}
\end{itemize}

\hr

\subsection{第五节、国防法规}
\subsubsection{现行法规介绍}
\paragraph{问:我国武装力量的\强调{性质和根本任务}由什么法律决定?}
\begin{itemize}
    \item 《中华人民共和国宪法》
\end{itemize}

\paragraph{《中华人民共和国兵役法》}
\begin{itemize}
    \item 第三条规定: (兵役制度)是 \textbf{\强调{志愿兵役为主体的 志愿兵役与义务兵役相结合的}}
    \item 第六条规定:(分类)兵役分为现役与预备役,前者->\textbf{军人};后者->\textbf{预备役人员}
    \item 第二十条规定:(年龄)征集服现役:
    \begin{enumerate}
        \item 对于\textbf{年满 \强调{18周岁}}的男性公民应被征集服现役;当年未被征集的,在\textbf{\强调{22周岁}}之前仍可被征集服现役 ;若高校毕业生 ,年龄则可以放宽至\textbf{\强调{24周岁}};若研究生,年龄则可放宽至\textbf{\强调{26周岁}}
        \item 对于女性公民 :(\textbf{\强调{根据军队需要}})同上
        \item (\textbf{根据军队需要和本人自愿})可征集年满\textbf{17周岁未满18周岁的公民服现役}
    \end{enumerate}
    \item 第二十七条规定:(如何选改为军士?)
    \begin{enumerate}
        \item \textbf{义务兵现役服满}
        \item \textbf{服现役期间表现优秀(根据军队需要和本人自愿)}
        \item \textbf{根据军队需要可以直接从非军士部门中具有专业技能的公民中招收}
    \end{enumerate}
    \item 军士实行分级服现役制度:\textbf{\强调{军士服役不超过30年,年龄不超过55周岁}}
\end{itemize}

\newpage

\section{第二章 \quad 国家安全}

\subsection{第一节、国家安全概述}
\subsubsection{总体国家安全观}
\textbf{\强调{对应!!!}}
\begin{itemize}
    \item \textbf{宗旨:人民安全}
    \item \textbf{根本:政治安全}
    \item \textbf{基础:经济安全}
    \item \textbf{保障:军事、文化、社会安全}
    \item \textbf{依托:促进国际安全}
\end{itemize}

\hr

\subsection{第二节、国家安全形势}
\begin{itemize}
    \item \textbf{邻国:\强调{朝鲜和越南}}既是海上又是陆上邻国
    \item \textbf{世界地缘:}可分为海洋地缘和亚欧大陆地缘战略区 ;我国\textbf{\强调{亚欧大陆地缘战略区}}
    \item \textbf{新兴领域的国家安全:}
    \begin{enumerate}
        \item \textbf{\强调{太空安全}}
        \item \textbf{\强调{深海安全}}
        \item \textbf{\强调{极地安全}}
        \item \textbf{\强调{海外利益安全}}
    \end{enumerate}
\end{itemize}

\hr

\subsection{第三节、国际战略形势}
\begin{itemize}
    \item \textbf{全球南方:}“全球南方”是什么?
    \begin{itemize}
        \item 是新兴市场国家和发展中国家的集合体
        \item \textbf{\强调{不是一个国际组织或政治集团}},没有明确的成员组成
        \item \textbf{特点:}包含多元价值观念、文化传统和发展水平;整体松散
    \end{itemize}
    \item \textbf{国际格局走向“多极化”} :(原因)\textbf{政治经济发展不平衡->多极力量均衡化趋势->两极体制解体->多极化}
    \item \textbf{中国对国际战略形势的影响}
    \begin{enumerate}
        \item 在反对霸权主义和强权政治上起制约作用
        \item 在经济发展上起示范作用
        \item 在维护第三世界权益的斗争中起重要作用
    \end{enumerate}
\end{itemize}

\hr

\subsection{第四节、构建人类命运共同体}
\begin{itemize}
    \item 坚持共建共享,建设一个普遍安全的世界:
\end{itemize}
\begin{enumerate}
    \item 各方面应该树立\textbf{\强调{共同、综合、合作、可持续}}的安全观 \textbf{(世界安全,不是一国独自安全)}
    \item 解决世界安全的总钥匙是什么? \textbf{发展}(就是最大的安全)
\end{enumerate}

\newpage

\section{第三章 \quad 军事思想}

\subsection{第一节、军事思想概述}
\subsubsection{基本特征}
\begin{enumerate}
    \item 鲜明的阶级性
    \item 强烈的时代性
    \item 明显的继承性
    \item 显著的实验性
\end{enumerate}

\hr

\subsection{第二节、外国军事思想}
\begin{itemize}
    \item \textbf{最早追溯到文艺复兴时期(16th century)}
    \item 代表:
    \begin{enumerate}
        \item \textbf{拿破仑:}重视武力与思想的双重作用;重视军队的改革和建设;十分重视歼灭战的作战原则;善于集中兵力、以少击多
        \item \textbf{克劳塞维茨(《战争论》):}
        \begin{enumerate}
            \item \textbf{战争是社会集团为了达到一定目的的暴力行动,是为了“迫使敌人服从我们的意志”}
            \item \textbf{政治是战争的母体:“战争无非是政治通过另一种手段的继续”}
            \item \textbf{战争的政治目的是消灭敌人,必然要武力决战}
            \item \textbf{进攻和防御是战争的两种基本作战形式}
        \end{enumerate}
    \end{enumerate}
\end{itemize}

\hr

\subsection{第三节、中国古代军事思想}
\begin{itemize}
    \item 中国古代军事思想载体是\textbf{兵书}
    \item 以\textbf{\强调{《孙子兵法》}} 为代表的兵书的诞生,标志着\textbf{\强调{中国古代军事思想的基本形成}}
    \item \textbf{\强调{宋朝}}编撰了《五经七书》(什么时候编纂的?)
    \item 《孙子兵法》
    \begin{enumerate}
        \item \textbf{将帅选用的标准:“将者,智、信、仁、勇、严也”}
        \item \textbf{战略战术思想:}以“全”争胜,不战而屈人之兵;强调进攻速胜,反对持久作战;奇正结合,出奇制胜
    \end{enumerate}
\end{itemize}

\hr

\subsection{第四节、中国当代军事思想}
\begin{itemize}
    \item \textbf{毛泽东军事思想:}
    \begin{enumerate}
        \item 是以毛泽东为代表的\textbf{中国共产党人集体智慧}的结晶 (\textbf{\强调{非独创!}})
        \item 该思想进入成熟期的标志:建国前夕,毛泽东指出\textbf{不但要有强大的陆军,还要有强大的空军和海军}
    \end{enumerate}
    \item \textbf{该思想的地位:}丰富发展了马列主义军理、是中国革命取得胜利和国防建设的理论指南
    \item \textbf{主要内容:}战争观和方法论、人民军队建设思想(三大民主:\textbf{政治、经济、军事民主})、人民战争思想
    \item \textbf{邓小平新时期军队建设思想}:当前世界大战\textbf{可以避免}、革命军队的三化(\textbf{革命化、现代化、正规化})
    \item \textbf{江泽民国防和军队建设思想}:历史性课题是\textbf{”打得赢、不变质“}
    \item \textbf{胡锦涛国防和军队建设思想}:贯彻落实\textbf{科学发展观},核心是\textbf{\强调{以人为本}}
    \item \textbf{习近平强军思想(注意这里是论述题的增分点,小题目没有所以我就略了)}:强军目标、军政
\end{itemize}

\newpage

\section{第四章 \quad 现代战争}

\subsection{第一节、战争概述}
\begin{itemize}
    \item 随着\textbf{\强调{人类社会的发展和科学技术的进步}},\textbf{\强调{战争形态}}将不断\textbf{\强调{演变}},但是\textbf{\强调{战争的本质(战争是政治的继续)不会改变。}}
    \item 理解战争的\textbf{\强调{起源与根源的区别}}:
    \begin{itemize}
        \item 起源:引发战争的\textbf{直接因素}(政权、领土、经济、民族、宗教等冲突)
        \item 根源:\textbf{产生于私有制和剥削阶级}
    \end{itemize}
\end{itemize}

\hr

\subsection{第二节、新军事革命}
\begin{itemize}
    \item $C^4$ISR系统:指挥 (Command), 控制 (Control), 通信 (Communication), 计算机 (Computer), 情报 (Intelligence), 监视 (Surveillance), 侦察 (Reconnaissance)
    \item \textbf{新军事组织体制}:军队规模缩小、结构优化、作战指挥“扁平网络化”、部队编制“小型化、一体化、多能化”
\end{itemize}

\hr

\subsection{第三节、机械化战争}
\paragraph{特点}:武器装备性能优良、作战力量增多、战场范围扩大、出现新样式、立体/纵深作战成为重要方式

\hr

\subsection{第四节、信息化战争}
\begin{itemize}
    \item \textbf{构成要素}:信息化武器装备(软/硬杀伤、新概念武器)、\textbf{信息战}(指挥控制战)、\textbf{制信息权}
    \item \textbf{基本特征}:信息资源主导化、战场空间多维化、要素一体化、指挥扁平化、持续时间短促化、样式多样化
\end{itemize}

\newpage

\section{第五章 \quad 信息化装备}

\subsection{第一节、信息化装备概述}
\begin{itemize}
    \item \textbf{武器(兵器)}:直接用于杀伤有生力量和破坏军事设施的器械与装置统称。
    \item \textbf{三大系统}:信息化主战装备、军事信息系统、信息化保障装备
    \item \textbf{趋势}:隐形化、多功能化、无人化
    \item \textbf{天战武器}:反导武器、反卫星武器、空天飞机、太空作战飞行器
\end{itemize}

\hr

\subsection{第二节、信息化作战平台}
\begin{itemize}
    \item \textbf{定义}:信息化武器及其载体的总称(坦克、火炮、飞机、舰艇等)。
    \item \textbf{影响}:方式改变、进程加快、空间扩大、突然性增大。
\end{itemize}

\hr

\subsection{第三节、综合电子信息系统}
\begin{itemize}
    \item \textbf{功能}:指挥控制、预警探测、通信导航、电子对抗、综合保障。
    \item \textbf{主要目标}:外层空间、大气层、水面/水下、陆上目标。
\end{itemize}

\hr

\subsection{第四节、信息化杀伤武器}
\begin{itemize}
    \item \textbf{精确制导武器}:命中率在50\%以上。
    \item \textbf{核武器}:原子弹(裂变)、氢弹(聚变)。杀伤因素包括:光辐射、冲击波、早期核辐射、核电磁脉冲、放射性沾染。
    \item \textbf{高能激光武器}:优点是速度快、精度高、机动灵活;缺点是受气候(云雾雨雪)影响大。
\end{itemize}

\end{document}